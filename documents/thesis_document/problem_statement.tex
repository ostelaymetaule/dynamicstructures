\chapter{Problem Statement}
\label{sec:problem}


%1. a concise statement of the question that your thesis tackles
The purpose of this thesis is to tackle a specific instance of the problem of planning connected architectural geometric elements into an environment with existing geometry which is to be used as the supporting structure for the architectural elements. We handle the specific case in which the pre-existing geometry is a virtual forest, using the trunks and branches of the trees as supporting structures for architectural elements such as platforms, bridges, stairs and buildings. We propose several heuristics for the generation of a connected 'tree house community' and using these heuristics we propose a planning algorithm. The architectural elements need a large degree of adjustibility to be able to incorporate them into the irregular environment posed by the tree geometry, therefore we present procedural methods for the generation of these elements. Another goal for this thesis was the design of intuitive procedural modeling user tools.      

This work has been inspired by advancements in the fields of procedural plant generation and procedural generation of urban structures like buildings, roads and entire cities. These two fields have been explored to some extent now. Although the generation of urban structures is a relatively new research area compared to procedural plant generation. Never before have these fields been combined for our idea: the generation of architectural geometry controlled and supported by tree-like geometry. With our research we attempt to 
generate interesting meaningful \emph{'tree-house community'} scenes.

%3. discussion of why it is worthwhile to answer this question.
I believe this is an interesting problem to be solved for the following reason. Previous solutions for procedural virtual worlds focussed on realistic synthesis of the real world around us. In games, virtual worlds are designed with high user-entertainment value as top priority. This also means that the configuration of geometry in virtual game worlds must enable the user to enjoy the playing experience. Determining the leveldesign principles that make a virtual world enjoyable is hard on its own. Transforming these principals into formal rulesets is even harder. To make the problem even more complex, the principles for good leveldesign highly depend on the game play rules the virtual world dictates. We present two scenarios that incorporate a set of gameplay rules. From these scenarios and known pricipals of good leveldesign we create heuristics. Using these heuristics we construct a planning algorithm that takes forest geometry, a set of architectural elements and parameters we extracted from the scenarios as input and outputs a geometric configuration of the architectural elements.     

An important inspiration for our research is work done on the procedural generation of dungeons. 

%Although popular techniques for creating plants (prezmyslaw) and buildings (Muller) show similarities, there are some important %differences.     