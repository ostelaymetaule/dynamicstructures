\chapter{Problem Statement}
\label{sec:problem}


%1. a concise statement of the question that your thesis tackles
The purpose of this thesis is to tackle a specific instances of the problem of planning connected architectural elements within pre-exisiting structure for the generation of meaningful game worlds. The pre-existing structure is in this case a schematic outlay of a forest. I define a forest as "a set of spatially distributed trees". I abstract the definition of a tree to be a structure that is capable of supporting architectural elements such as stairs, platforms, buildings and bridges connected to other trees. 
Tranforming the forest environment to a graph problem, the stais become edges that connect platform nodes within a single tree and bridges form other type of edges between platform nodes on different branches that roughly reside on the same height. The actual geometry of trees in this forest is only generated in the final step of the method since it is not included as a requirement for the planning of architectural structures. 

Since world structures of certain types of succesfull virtual game worlds often show repeated patterns that are needed for good gameplay we believe that gameworlds are interesting to describe algorithmically. We will analyse three types of world structure scenarios which are based on leveldesign of very established game genres: 

\begin{itemize}
\item Dungeon Adventure Maps 
\item Stealth / Hide and Seek Maps 
\item First Person Shooter Multiplayer Maps 
\end{itemize} 

From each scenario we extract several heuristics for the planning of game level structure and using these heuristics we propose planning algorithms for each scenario. An additional step is the generation of forest and architecture geometry.  

The architectural elements need a large degree of adjustibility to be able to incorporate them into the irregular environment posed by the forest structure, therefore we present procedural methods for the generation of these elements. Another goal for this thesis was the design of intuitive procedural modeling user tools.      

This work has been inspired by advancements in the fields of procedural plant generation and procedural generation of urban structures like buildings, roads and entire cities. These two fields have been explored to some extent now. Although the generation of urban structures is a relatively new research area compared to procedural plant generation. Never before have these fields been combined for our idea: the generation of architectural structure within a group of tree-like structures. With our research we attempt to generate logical game map structures for the proposed scenarios.

%3. discussion of why it is worthwhile to answer this question.
I believe this is an interesting problem to be solved for the following reason. Previous solutions for procedural virtual worlds focussed on realistic synthesis of the real world around us. In games, virtual worlds are designed with high user-entertainment value as top priority. This also means that the configuration of geometry in virtual game worlds must enable the user to enjoy the playing experience. Determining the leveldesign principles that make a virtual world enjoyable is hard on its own. Transforming these principles into formal rulesets is non trivial. We must also consider that the movement restrictions and abilities that are defined by the gameplay have major influence on the effectiveness of the design of a game-world. 

%We present two scenarios that incorporate a set of gameplay rules. From these scenarios and known pricipals of good leveldesign we %create heuristics. Using these heuristics we construct a planning algorithm that takes forest geometry, a set of architectural elements %and parameters we extracted from the scenarios as input and outputs a geometric configuration of the architectural elements.     

An important inspiration for my research is work done on procedural generation of virtual game worlds in games such as "Love" and "Diablo 2". 

%Although popular techniques for creating plants (prezmyslaw) and buildings (Muller) show similarities, there are some important %differences.     
