\chapter{Problem Statement}
\label{sec:problem}


%1. a concise statement of the question that your thesis tackles
The purpose of this thesis is to tackle a specific instance of the problem of planning connected architectural geometric elements into an environment with existing geometry which is to be used as the supporting structure for the architectural elements. We handle the specific case in which the pre-existing geometry is a virtual forest, using the trunks and branches of the trees as supporting structures for architectural elements such as platforms, bridges, stairs and buildings. We propose several heuristics for the generation of a connected 'tree house community' and using these heuristics we propose a planning algorithm. The architectural elements need a large degree of adjustibility to be able to incorporate them into the irregular environment, therefore we present procedural methods for the generation of these elements.       

%We looked at real-world examples of tree-based architecture but we did not limit the properties of  

%2. justification, by direct reference to section 3, that your question is previously unanswered




%3. discussion of why it is worthwhile to answer this question.

