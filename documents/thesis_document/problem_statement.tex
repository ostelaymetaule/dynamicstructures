\chapter{Problem Statement}
\label{sec:problem}
%\desc{
%Engineering theses tend to refer to a "problem" to be solved where other disciplines talk in terms of a "question" to be answered. In either case, this section has three main parts:
%1. a concise statement of the question that your thesis tackles
%2. justification, by direct reference to section 3, that your question is previously unanswered
%3. discussion of why it is worthwhile to answer this question.
%Item 2 above is where you analyze the information which you presented in Section 3. For example, maybe your problem is to "develop a Zylon algorithm capable of handling very large scale %problems in reasonable time" (you would further describe what you mean by "large scale" and "reasonable time" in the problem statement). Now in your analysis of the state of the art you %would show how each class of current approaches fails (i.e. can handle only small problems, or takes too much time). In the last part of this section you would explain why having a %large-scale fast Zylon algorithm is useful; e.g., by describing applications where it can be used.
%Since this is one of the sections that the readers are definitely looking for, highlight it by using the word "problem" or "question" in the title: e.g. "Research Question" or "Problem %Statement", or maybe something more specific such as "The Large-Scale Zylon Algorithm Problem." 
%}


%1. a concise statement of the question that your thesis tackles
The purpose of this thesis is to tackle a specific instance of the problem of planning connected architectural geometric elements into an environment with existing geometry which has to be used as the supporting structure for the architectural elements. We handle the specific case in which the pre-existing geometry is a virtual forest, using the trunks and branches of the trees as supporting structures for architectural elements such as platforms, bridges, stairs and buildings. We propose several heuristics for the generation of a connected 'tree house community' and using these heuristics we construct a procedural algorithm. The architectural elements need a large degree of adjustibility to be able to incorporate them into the irregular environment, therefore we present procedural methods for the generation of these elements.       

%We looked at real-world examples of tree-based architecture but we did not limit the properties of  

%2. justification, by direct reference to section 3, that your question is previously unanswered




%3. discussion of why it is worthwhile to answer this question.

