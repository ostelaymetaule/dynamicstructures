
\chapter{Introduction}
\label{sec:intro}

%This is a general introduction to what the thesis is all about -- it is not just a description of the contents of each section. Briefly summarize the question (you will be stating the %question in detail later), some of the reasons why it is a worthwhile question, and perhaps give an overview of your main results. This is a birds-eye view of the answers to the main %questions answered in the thesis.
Procedural content generation (pcg) lately received a lot of attention from various players in the field. It is believed that PCG will eventually play a major role in virtual worlds of simulations and games, replacing a significant amount of conventional man-made content, and thereby reducing costs and improving time efficiency.  

This thesis adresses the problem of procedural architecture in the special case of restrictions and possibilities enforced by the geometric properties of pre-existing objects, trees, in the scene. A side goal for this research was to develop methods for intuitive interactive control over the objects wich were to be generated. 

%Regarding user control and procedural techniques it is important to strike a balance which allows for a fast %and intuitive building process, and meanwhile the user should have just the right amount of control to make the %scene to its liking.

%Rewriting systems, l-systems waarna shape grammars volgen
%The Shape grammar concept was originally coined by Stiny in his much cited \emph{"Shape and Shape Grammar"} \citep{Stiny80}.
%<hier de description van een shape>

Visual modeling of plant development is a field which started in 1962, when Ulam applied cellular automata to 
simulate the development of branching patterns \citep{PrzemyslawPlants}. A formalism for modeling plants was proposed by Lindenmayer 
in 1968, this formalism was called L-systems since. The following definition of an L-system is given by Przemyslaw \citep{PrzemyslawPlants}: 

\begin{quote}
An L-system is a parallel rewriting system operating on branching structures represented as bracketed strings of symbols with associated parameters, called modules. Matching pairs of square brackets enclose branches. Simulation begins with an  initial string called the axiom, and proceeds in a sequence of discrete deriviation steps. In each step, rewriting rules or productions replace all modules on the predeccesor string by succesor modules.   
\end{quote}   

Przemyslaw \citep{PrzemyslawPlants} has employed and extended the l-system formalism for realistic visualisation of entire plant ecosystems. Within a ecosystem organisms interact with each other and this interaction determines many properties for individual organisms; such as growth rate. Since the original L-system formalism does not account for communication between two processes, Przemyslaw proposed \emph{open l-systems} which incorporates \emph{communication modules}.    
  
Since 2001 L-systems also proved to be useful in the field of urban procedural generation \citep{Wonka03}. It was then that Muller proposed the use of L-systems for the generation of road networks and building generation. 

%hier wat over building, generation, city generatoin, texture generation.
%shape grammars

Techniques for tree generation in this thesis are based on L-systems, however these techniques are simplified to a certain degree since the modeling of plants is not the maintopic of this thesis. We have used the l-system formalism to generate simple branching tree structures. We did not strive to generate visually realistic models of trees since this has already been achieved by many people before me with impressive results. Instead our simplified method generates the main structure of a tree which is then used as input in our method for the generation of treehouse architecture. 

%<hier resultaat van treehouse generatie module>

It is often the case that traditional general purpose modeling software is hard to master as a result of the freedom which is given to the user. Special purpose modeling tools such as cityEngine \citep{Muller06} (in the case of urban modeling) and speedTree (in the case of modeling plants and trees) provide procedural methods to generate objects within a certain class with very high time effiency. However the human touch still remains a very important part of the modeling procedure. Eastatics are very hard to turn into a set of formal rules on which an algorithm can operate.

The special purpose modeling tool for the generation of organic geometry and treehouse architecture that was developed for this thesis provides the user with intuitive interaction tools to allow easy manipulation.        


%\old{   
%The research which preceded this document strived to construct an intuitive organic modelling method for the creation of 2D maps. The central focus of this research has been on methods %for construction of dynamic structures using real-time growth models. To enable a large degree of control over these dynamic structures the is a need for skeletons. A large part of this %thesis is about constructing skeleton representations for dynamic structures. There obviously is a realtime demand in order to enable realtime interactivity, which means building such %skeletons should above all be an efficient process.    
%}

\section{Overview}
\label{subsec:overview}

This thesis is structured as follows. In the next section I will discuss related work. In the related work section I will bring my thesis into context with regard to procedural modeling of trees and architecture. A precise statement of the problem and why it's an interesting problem to solve will follow in section \ref{sec:problem}. 

In section \ref{sec:concept} I will present the conceptual system model. The method for the generation of the forest layout and the generation of the tree geometry is presented in section \ref{sec:pfg}.  
 
With the forest geometry in place we are ready to review the planning method which is responsible for the construction of a connected graph representing a tree community. I will conclude the method description with a look at the final stage of the pipeline, translating the symbols from the nodes and edges of the graph to the geometrical architectural elements. Section \ref{sec:UIT} discusses user interactivity and usability of the proposed system. The last two sections present results and conclusions respectively.     
