\documentclass{article}

\usepackage[usenames]{color} 
\usepackage{graphicx} 

\definecolor{lightblue}{rgb}{0.0, 0.8,1.0} 
\definecolor{myOrange}{rgb}{1.0, 0.6,0.0} 
\definecolor{myGrey}{rgb}{0.1, 0.1,0.1} 
\definecolor{MyDarkGreen}{rgb}{0.13, 0.54,0.13} 


\newcommand{\todo}[1]{\textcolor{red}{\textbf{\newline TODO: }\it{#1} \newline}}
\newcommand{\inhoud}[1]{\textcolor{blue}{\textbf{\newline Summary: }\it{#1}}}
\newcommand{\revise}[1]{\textcolor{myOrange}{\textbf{\newline Revise: }\it{#1}}}
\newcommand{\old}[1]{\textcolor{myGrey}{\small{\textbf{\newline Old: }\it{#1}}}}
\newcommand{\desc}[1]{\textcolor{lightblue}{\textbf{\newline description: }\it{#1} \newline}}

\newcommand{\voegtoe}[1]{\textcolor{MyDarkGreen}{\textbf{-Insert: }\it{#1}}}

\title{Interactive Procedural Modeling: Architecture in Organic Scenes  \small{interactivity in procedural modeling tools}}
\author{Ruud op den Kelder}

\begin{document}

\maketitle

\begin{abstract}
Interactive Procedural Modeling: Architecture in Organic Scenes 
\end{abstract}
\newpage 

\tableofcontents
\newpage 

\section{Introduction}

\desc{This is a general introduction to what the thesis is all about -- it is not just a description of the contents of each section. Briefly summarize the question (you will be stating the question in detail later), some of the reasons why it is a worthwhile question, and perhaps give an overview of your main results. This is a birds-eye view of the answers to the main questions answered in the thesis.}

This thesis is about procedural architecture in the special case of restrictions and possibilities enforced by the geometric properties of
pre-existing objects, trees, in the scene. A side quest for this thesis was to develop methods for intuitive interactive control over the 
objects wich were to be generated. Regarding user control and procedural techniques it is important to strike a balance which allows for a 
fast and intuitive building process, and meanwhile the user should have just the right amount of control to make the scene to its liking. 



     






User control over the growing polygons is provided with several handles: 
\revise{maak dit beter...}

\begin{enumerate}
\item a dynamic innerskeleton which is formed by the contour of the polygon.   
\item the vertices of the polygon contour can be dragged.
\item scripting properties of the growing surface.
\item global properties
\end{enumerate}

\todo{arguments in favour of this method: ease of use. experimental curiousity}

The research which preceded this document strived to construct an intuitive organic modelling method for the creation of 2D maps. The central focus of this research has been on methods for construction of dynamic structures using real-time growth models. To enable a large degree of control over these dynamic structures the is a need for skeletons. 
A large part of this thesis is about constructing skeleton representations for dynamic structures. There obviously is a realtime demand in order to enable realtime interactivity, which means building such skeletons should above all be an efficient process.    

\desc{\textbf{2. Background Information (optional)}
A brief section giving background information may be necessary, especially if your work spans two or more traditional fields. That means that your readers may not have any experience with some of the material needed to follow your thesis, so you need to give it to them. A different title than that given above is usually better; e.g., "A Brief Review of Frammis Algebra." 
}

\section{Review of State of the Art in Procedural Modeling of urban geometry and organic geometry}

\desc{
3. Review of the State of the Art
Here you review the state of the art relevant to your thesis. Again, a different title is probably appropriate; e.g., "State of the Art in Zylon Algorithms." The idea is to present (critical analysis comes a little bit later) the major ideas in the state of the art right up to, but not including, your own personal brilliant ideas.
You organize this section by idea, and not by author or by publication. For example if there have been three important main approaches to Zylon Algorithms to date, you might organize subsections around these three approaches, if necessary:
3.1 Iterative Approximation of Zylons
3.2 Statistical Weighting of Zylons
3.3 Graph-Theoretic Approaches to Zylon Manipulation 
}

procedural techniques: 
* l-systems
* Extended l-systems
* Shape Grammar
* Scattering techniques

\subsection{Procedural Modeling}
\inhoud{beschrijf kort geschiedenis van procedural modeling en werk naar map generation}

Procedural modeling                        


\subsection{Map generation}
\inhoud{beschrijf huidige games en tools die gebruik maken van procedural generation voor geometrie van virtuele werelden.}

\subsubsection{spatial growth based}
\inhoud{verhaal over het feit dat spatial growth based-user controlled bouwen van maps relatief nieuw is.}

\subsubsection{Interactivity}
\inhoud{beschrijf tools voor het manipuleren van modellen die afhankelijk zijn van l-systems en tools die worden 
	gebruikt voor level design, zoals tools voor terrain editors.}


\subsection{Required Input}
\todo{Belangrijk om snel vast te stellen hoeveel en welke input er nodig is.}


\subsection{The growth model map generator tool}
\todo{discusseren dat gebruiker gelimiteerde controle heeft over het proces: nadelen en voordelen}

   
\section{The problem: User control in procedural generation}

\desc{
Engineering theses tend to refer to a "problem" to be solved where other disciplines talk in terms of a "question" to be answered. In either case, this section has three main parts:
1. a concise statement of the question that your thesis tackles
2. justification, by direct reference to section 3, that your question is previously unanswered
3. discussion of why it is worthwhile to answer this question.
Item 2 above is where you analyze the information which you presented in Section 3. For example, maybe your problem is to "develop a Zylon algorithm capable of handling very large scale problems in reasonable time" (you would further describe what you mean by "large scale" and "reasonable time" in the problem statement). Now in your analysis of the state of the art you would show how each class of current approaches fails (i.e. can handle only small problems, or takes too much time). In the last part of this section you would explain why having a large-scale fast Zylon algorithm is useful; e.g., by describing applications where it can be used.
Since this is one of the sections that the readers are definitely looking for, highlight it by using the word "problem" or "question" in the title: e.g. "Research Question" or "Problem Statement", or maybe something more specific such as "The Large-Scale Zylon Algorithm Problem." 
}

\subsection{} 
\revise{bedenk een betere titel}

\subsection{User control} 

\todo{elke van de volgende controls behandelen met behulp van bestaande tools die hier gebruik van maken}


\subsubsection{local control}


\subsubsection{global control}


\subsection{Growing surfaces: An alternative method for area generation}

\section{Using realtime dynamically growing surfaces}

\subsection{dynamic structures: polygonal surfaces}
\inhoud{2D polygonale expansie van een ondergrond model als land of water die wordt gestuurd door eigenschappen van het type maar ook door ruimtelijke manipulatie methodes zoals vectorfields. representatie(onder voorbehoud): 
\begin{itemize}
\item Cellular automata
\item Cell systems
\item iets wat niet met cellen werkt  
\end{itemize}
realtime hertriangulatie zal moeten worden toegepast. 
polygonal meshes: 
\begin{itemize}
\item representation for irregular 2D volumes.  
\item suitable for water, land, cave like structures  
\item real-time demand, so need for efficient algorithms
\item houdt rekening met triangulatie
\item houdt rekening met smoothing
\item datastructure
\end{itemize}
citeer: On Vertex-Vertex Systems and Their Use in
Geometric and Biological Modelling
}

\todo{dit kan al snel worden geimplementeerd dus logisch om dit als eerste te beschrijven en methode uit te werken.}


\subsection{Surfaces}

\subsection(Forest Generation Scattering techniques)

\subsection(Forest Generation Scattering techniques)

\inhoud{Per type spatial growth model bespreken wat de context is binnen outdoor areas en dus met 
welke andere spatial growth modellen het communiceert en welke effecten het spatial growth model heeft 
op de andere en vica versa. Ook bespreken op wat voor manier een vector field (en andere tools 
die invloed hebben op de ruimtelijke indeling)  effect heeft op de uitdijing van dit type model.   
}

A definition of land: "The part of Earth which is not covered by oceans or other bodies of water".(bron: http://en.wiktionary.org/wiki/land ...jaja ik vind nog wel een betere bron)
In reality land obviously does not grow, so first we must ask ourselves whether growing land in our virtual world can be made useful and intuitive to use. There are no real rules for the bounding shape of a piece of land, the bounds can be configured in any way. However due to this fact almost any random configuration of the shape of land looks realistic. The question is, as with other growth models with a minimum amount of rules, whether generating land by means of a growth model, thereby having less control over the result compared to conventional methods, is something which we want. 

When you want maximum control over the resulting geometry of a piece of land, using a growth model is not a very attractive method.However there are some situations in which this method is quite helpful.

\inhoud
{wanneer is het groeien van land handig: 
\begin{itemize}
\item Obviously: In situaties waarin de gebruiker niet veel waarde hecht aan de precieze vorm van de resulterence geometrie.  
\item stimuleert de creativiteit: elaborate. 
\end{itemize}
}

Land provides the underlying surface of many of the growth models discussed in this paper. 

\inhoud
{dus het bestaan van land op de seedpositie van een afhankelijk groeiproces is een eis. Wellicht moet het genereren van een map bestaan uit verschillende fases:
\begin{itemize}
\item fase 1: generatie van land en water. 
\item fase 2: plaatsen en activeren van groei processen die afhankelijk zijn van land.  
\end{itemize}
}

The field of biological modeling provides some interesting modeling techniques for the process of cell division. 
Cell division is interesting with respect to the concept of growing land because it is an expanding process with some nice properties. \voegtoe{benoem properties of revise deze zin.} 


\subsubsection{Water} 


\subsubsection{Afforestation} 

\subsubsection{Caves}

\section{Organic map generation tool}

\subsection{Modifiers}

\subsubsection{Spatial modifiers}

\inhoud{
\begin{itemize}
\item vector fields: dit moet je wellicht helemaal aan het begin doen omdat alle spatial growth models
gemanipuleerd worden door deze vector fields.
\item obstruction with solid objects
\item smudge-like tool
\end{itemize}
}

\subsection{User controlled map generation}

\subsection{automatic map generation} 


\section{Implementation}
\subsection{Tools used}
\inhoud{ 
\begin{itemize}
\item programming language: c++ (for now)
\item visualization library: openGL or Ogre3D (if it features easy vertex manipulation) 
\item physics engine: ODE (onder voorbehoud)  
\item collision detection: D-Collide (onder voorbehoud)
\end{itemize}
}

\subsection{Application Class Structure}

\subsection{User Interface}
\subsection{Communication system between models}

\section{Results}

\section{Conclusion}

\newpage 
\bibliographystyle{abbrv}	% (uses file "plain.bst")
\bibliography{thesisrefs}	% expects file "myrefs.bib"

\end{document}

