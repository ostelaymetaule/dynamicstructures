

\chapter{Procedural Forest Generation}
\label{sec:pfg}



\section{Scattering Techniques for Tree Positions}
\label{subsec:scattering}

\citet{PrzemyslawPlants}


\citet{}

\citet{mickwestgamasutra}
%\begin{algorithm}
%
%\caption{this is the pseudocode for the spatial layout of the trees}
%\begin{algorithmic}
%\IF {$i\geq maxval$} 
%        \STATE $i\gets 0$
%\ELSE
%        \IF {$i+k\leq maxval$}
%                \STATE $i\gets i+k$
%        \ENDIF
%\ENDIF 
%\end{algorithmic}

%\end{algorithm}

\section{Ecosystem Modeling}


\section{Simple Tree Structures}

Procedurally generating trees by means of l-systems has had a great amount of succes since the original proposal by Lindenmayer. The L-system formalism is widely applied in  academic work and is also succesfully used within commercial applications. This thesis does not focus on the l-system formalism in particular, since it is already a well established theory \citet{PrzemyslawAlgoBeauty}. However, the following section will describe the basics of l-systems to establish some basic understanding wich will be needed in following sections. 

%l-system uitleg hier 

I am interested in the structure of trees and the possibilities and restrictions it poses for placement of 
architectural shapes. For the purpose of this thesis the generated trees do not have to be visually convincing, 
however the basic shape should still be identified as a tree. The geometric properties of a tree model that are to interest of us are those that have effect on the possibilities with respect to the incorporation of architectural man-made structures. The tree geometry functions as the support structure for the building blocks I define in detail in section \ref{sec:treehousearch}. 

%identify geometrically suitable foundations
In this section we will identify geometric configurations within tree geometry that allow for the construction of the proposed set of architectural elements.    

%algorithm for tree generation: 
%		-niet te veel sub-branches
%   
%
%

