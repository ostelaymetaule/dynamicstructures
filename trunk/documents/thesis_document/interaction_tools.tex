
\chapter{User Interaction Tools}
\label{sec:UIT}

\section{Growing Surfaces}
\label{sec:GrowingSurfaces}

%Per type spatial growth model bespreken wat de context is binnen outdoor areas en dus met 
%welke andere spatial growth modellen het communiceert en welke effecten het spatial growth model heeft 
%op de andere en vica versa. Ook bespreken op wat voor manier een vector field (en andere tools 
%die invloed hebben op de ruimtelijke indeling)  effect heeft op de uitdijing van dit type model.   


%A definition of land: "The part of Earth which is not covered by oceans or other bodies of water".(bron: %http://en.wiktionary.org/wiki/land ...jaja ik vind nog wel een betere bron)
%In reality land obviously does not grow, so first we must ask ourselves whether growing land in our virtual %world can be made useful and intuitive to use. There are no real rules for the bounding shape of a piece of %land, the bounds can be configured in any way. However due to this fact almost any random configuration of the %shape of land looks realistic. The question is, as with other growth models with a minimum amount of rules, %whether generating land by means of a growth model, thereby having less control over the result compared to %conventional methods, is something which we want. 

%When you want maximum control over the resulting geometry of a piece of land, using a growth model is not a %very attractive method.However there are some situations in which this method is quite helpful.

%\inhoud
%{wanneer is het groeien van land handig: 
%\begin{itemize}
%\item Obviously: In situaties waarin de gebruiker niet veel waarde hecht aan de precieze vorm van de resulterence geometrie.  
%\end{itemize}
%}

%Land provides the underlying surface of many of the growth models discussed in this paper. 

%\inhoud
%{dus het bestaan van land op de seedpositie van een afhankelijk groeiproces is een eis. Wellicht moet het genereren van een map bestaan uit verschillende fases:
%\begin{itemize}
%\item fase 1: generatie van land en water. 
%\item fase 2: plaatsen en activeren van groei processen die afhankelijk zijn van land.  
%\end{itemize}
%}

%The field of biological modeling provides some interesting modeling techniques for the process of cell division. 
%Cell division is interesting with respect to the concept of growing land because it is an expanding process with some nice properties. \voegtoe{benoem properties of revise deze zin.} 

%\inhoud{
%\begin{itemize}
%\item vector fields: dit moet je wellicht helemaal aan het begin doen omdat alle spatial growth models
%gemanipuleerd worden door deze vector fields.
%\item obstruction with solid objects
%\item smudge-like tool
%\end{itemize}
%}

