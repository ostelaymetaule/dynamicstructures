
\chapter{State of the Art in Procedural Modeling}
\label{sec:relatedwork}

%plants 

\section{Plants and Trees}
Algorithmic Beauty of Plants by Lindenmayer and Przemyslaw \citep{PrzemyslawAlgoBeauty} is the first complete work discussing the generation of plant geometry using procedural methods such as L-systems and fractals. To model and visualise realistic ecosystems, Przemyslaw extended the l-systems concept to a system which allowed communication between systems (open l-systems \citep{PrzemyslawPlants} \citep{Deussen98}). Visual editing of procedural plants models is discussed in \citet{interactivebonsai}. 

%road networks
\section{Road Networks}
The foundation for procedural city and building modeling was provided by Parish and Muller \citet{Parish01} in their paper \emph{"Procedural Modeling of Cities"}. The main contribution of this paper is the use of extended L-systems for the generation of city roadmaps. They also propose a method for the texturing of facades. An intuitive editing approach for road networks with the use of tensor fields and bush techniques is presented by Chen et al. \citet{Chen08}. 

%building architecture
\section{Architecture}
An attempt was made to use L-systems for the creation of buildings \citet{Parish01}, however this did not prove to be effective. L-systems are designed to handle growth-like processes, it has been acknowledged that the construction process of a building is not a growth like process. Instead, building construction is better expressed by series of partitioning steps. These partitioning steps can be described by another kind of rewriting grammar called \emph{set grammar}. In \citet{Wonka03} Wonka presents a method for the automatic creation of building using such grammar systems. In this work Wonka introduces the idea of a specialized type of set grammar called \emph{split grammar} which operates on shapes. In \citet{Muller06} the split rules from the split grammar concept are defined in a grammar system called \emph{CGA Shape}, which was the first procedural system for the creation of detailed buildings with consistent mass 
models. The process of creating a ruleset in CGA shape for a specific type of building is not straightforward and requires a trained expert. Lipp et al. \citet{Lipp08} introduce a visual method for the editing of the CGA Shape grammar for procedural architecture to simplify the rule building process.

\section{Procedural Game Worlds}
%content here%

